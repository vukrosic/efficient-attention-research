\section{Kimi Delta Attention: Improving Delta Rule with Fine-grained Gating}
We propose Kimi Delta Attention (KDA), a new gated linear attention variant that refines GDN's scalar decay by introducing a fine-grained diagonalized gate $\brickred{\operatorname{Diag}(\boldsymbol{\alpha}_t)}$ that enables fine-grained control over memory decay and positional awareness (as discussed in \S\ref{sec:delta_rule}). We begin by introducing the chunkwise parallelization of KDA, showing how a series of rank-1 matrix transformations can be compressed into a dense representation while maintaining stability under diagonal gating. We then highlight the efficiency gains of KDA over the standard DPLR (\emph{Diagonal-Plus-Low-Rank}) formulation \cite{gu-2022-efficiently,peng-2025-rwkv7}.
\begin{equation}
    \mathbf{S}_t = \left(\mathbf{I}-\beta_t\bm{k}_{t}\bm{k}_{t}^{\top}\right)\brickred{\operatorname{Diag}\left(\bm{\alpha}_t \right)}\mathbf{S}_{t-1} + \beta_t\bm{k}_{t}\bm{v}_{t}^{\top}\in\mathbb{R}^{d_k\times d_v};
    \qquad \bm{o}_t = \mathbf{S}^\top_t \bm{q}_t\in\mathbb{R}^{d_v}
    \label{eq:recurrent_KDA}
\end{equation}
\definecolor{kvcolor}{RGB}{241,140,74}
\begin{figure*}[h!]
    \centering
    \begin{adjustbox}{width=0.9\textwidth}
        \begin{tikzpicture}
            \tikzset{
                state/.style={
                    matrix of nodes,
                    nodes={
                        minimum size=10pt,
                        anchor=center,
                        inner sep=0pt,
                        font=\tiny
                    },
                    nodes in empty cells,
                    inner sep=0pt,
                    rounded corners=1pt
                },
                statelink/.style={
                    dash pattern={on 4pt off 4pt},
                    ->,
                    >={Straight Barb[length=5pt, width=6pt]},
                    line width=1.3pt,
                    shorten >=8pt,
                    shorten <=8pt,
                },
                seplink/.style={
                    dash pattern={on 4pt off 4pt},
                    line width=1pt,
                    shorten >=8pt,
                    shorten <=8pt,
                },
                kvlink/.style={
                    ->,
                    >={Straight Barb[length=5pt, width=6pt]},
                    line width=1.3pt,
                    shorten >=8pt,
                    shorten <=8pt,
                },
            }
            \matrix[state, anchor=west, left delimiter={[}, right delimiter={]}] (st) {
            |[fill=white]| {} & |[fill=midnightblue!20]| {} & |[fill=midnightblue!10]| {} & |[fill=midnightblue!20]| {} \\
            |[fill=midnightblue!60]| {} & |[fill=midnightblue!70]| {} & |[fill=white]| {} & |[fill=midnightblue!40]| {} \\
            |[fill=white]| {} & |[fill=midnightblue!30]| {} & |[fill=midnightblue!50]| {} & |[fill=white]| {} \\
            |[fill=midnightblue!50]| {} & |[fill=midnightblue!20]| {} & |[fill=midnightblue!20]| {} & |[fill=midnightblue!50]| {} \\
            };

            \node[right=10pt of st] (eq) {\large $=$};
            \node[right=0pt of eq.east] (leftbracket) {\large $\Biggl($};
            \matrix[state, right=5pt of leftbracket, left delimiter={[}, right delimiter={]}] (identity) {
            |[fill=black!50]| {} & |[fill=white]| {} & |[fill=white]| {} & |[fill=white]| {} \\
            |[fill=white]| {} & |[fill=black!50]| {} & |[fill=white]| {} & |[fill=white]| {} \\
            |[fill=white]| {} & |[fill=white]| {} & |[fill=black!50]| {} & |[fill=white]| {} \\
            |[fill=white]| {} & |[fill=white]| {} & |[fill=white]| {} & |[fill=black!50]| {} \\
            };
            \node[right=10pt of identity] (minus) {\large $-$};

            \matrix[state,right=10pt of minus] (kvec)
            {
            |[fill=kvcolor!30]| {} \\
            |[fill=kvcolor!50]| {} \\
            |[fill=kvcolor!70]| {} \\
            |[fill=kvcolor!90]| {} \\
            };

            \node[anchor=west, inner sep=0pt] at (kvec.east) {\large $\times$};
            \matrix[state,
            anchor=north west, baseline=0.3cm
            ]
            (kt) at ($(kvec.north east)+(10pt,0)$) {
            |[fill=kvcolor!30]| {} & |[fill=kvcolor!50]| {} & |[fill=kvcolor!70]| {} & |[fill=kvcolor!90]| {} \\
            |[fill=white, opacity=0, minimum size=0.05cm]| {} & |[fill=white, opacity=0, minimum size=0.05cm]| {} & |[fill=white, opacity=0, minimum size=0.05cm]| {} & |[fill=white, opacity=0, minimum size=0.05cm]| {} \\
            };
            \node[right=50pt of kvec] (rightbracket) {\large $\Biggl)$};



            \matrix[state, right=5pt of rightbracket, left delimiter={[}, right delimiter={]}] (decay) {
            |[fill=brickred!80]| {} & |[fill=white]| {} & |[fill=white]| {} & |[fill=white]| {} \\
            |[fill=white]| {} & |[fill=brickred!70]| {} & |[fill=white]| {} & |[fill=white]| {} \\
            |[fill=white]| {} & |[fill=white]| {} & |[fill=brickred!50]| {} & |[fill=white]| {} \\
            |[fill=white]| {} & |[fill=white]| {} & |[fill=white]| {} & |[fill=brickred!60]| {} \\
            };

            \matrix[state, right=20pt of decay, left delimiter={[}, right delimiter={]}] (stm1) {
            |[fill=midnightblue!20]| {} & |[fill=midnightblue!40]| {} & |[fill=midnightblue!60]| {} & |[fill=white]| {} \\
            |[fill=midnightblue!90]| {} & |[fill=midnightblue!70]| {} & |[fill=midnightblue!20]| {} & |[fill=midnightblue!70]| {} \\
            |[fill=midnightblue!30]| {} & |[fill=midnightblue!40]| {} & |[fill=midnightblue!60]| {} & |[fill=midnightblue!50]| {} \\
            |[fill=midnightblue!10]| {} & |[fill=white]| {} & |[fill=midnightblue!80]| {} & |[fill=midnightblue!20]| {} \\
            };

            \node[right=10pt of stm1] (plus) {\large $+$};
            \matrix[state,right=10pt of plus] (kvec)
            {
            |[fill=kvcolor!30]| {} \\
            |[fill=kvcolor!50]| {} \\
            |[fill=kvcolor!70]| {} \\
            |[fill=kvcolor!90]| {} \\
            };

            \node[anchor=west, inner sep=0pt] at (kvec.east) {\large $\times$};
            \matrix[state,
            anchor=north west, baseline=0.3cm
            ]
            (kt) at ($(kvec.north east)+(10pt,0)$) {
            |[fill=kvcolor!30]| {} & |[fill=kvcolor!50]| {} & |[fill=kvcolor!70]| {} & |[fill=kvcolor!90]| {} \\
            |[fill=white, opacity=0, minimum size=0.05cm]| {} & |[fill=white, opacity=0, minimum size=0.05cm]| {} & |[fill=white, opacity=0, minimum size=0.05cm]| {} & |[fill=white, opacity=0, minimum size=0.05cm]| {} \\
            };
            
            \draw[seplink] ($(kvec.north)+(70pt,16pt)$) -- ($(kvec.south)+(70pt,-16pt)$);
            \matrix[state,right=80pt of kvec] (o)
            {
            |[fill=kvcolor!30]| {} \\
            |[fill=kvcolor!50]| {} \\
            |[fill=kvcolor!70]| {} \\
            |[fill=kvcolor!90]| {} \\
            };
            
            \node[right=5pt of o] (eq) {\large $=$};
            \matrix[state, anchor=west, right=10pt of eq, left delimiter={[}, right delimiter={]}] (st) {
            |[fill=white]| {}           & |[fill=midnightblue!60]| {} & |[fill=white]| {}           & |[fill=midnightblue!50]| {} \\
            |[fill=midnightblue!20]| {} & |[fill=midnightblue!70]| {} & |[fill=midnightblue!30]| {} & |[fill=midnightblue!40]| {} \\
            |[fill=midnightblue!10]| {} & |[fill=white]| {}           & |[fill=midnightblue!50]| {} & |[fill=midnightblue!20]| {} \\
            |[fill=midnightblue!20]| {} & |[fill=midnightblue!20]| {} & |[fill=white]| {}           & |[fill=midnightblue!50]| {} \\
            };
            
            \matrix[state,right=10pt of st] (q)
            {
            |[fill=kvcolor!30]| {} \\
            |[fill=kvcolor!50]| {} \\
            |[fill=kvcolor!70]| {} \\
            |[fill=kvcolor!90]| {} \\
            };

        \end{tikzpicture}

    \end{adjustbox}
    \captionsetup{labelformat=empty,labelsep=none}
    \label{fig:KDA-recurrent}
\end{figure*}





\subsection{Hardware-Efficient Chunkwise Algorithm}
\label{sec:kda:chunk}



By partially expanding the recurrence for Eq. \ref{eq:recurrent_KDA} into a chunk-wise formulation, we have:
\begin{equation}
    \begin{aligned}
        \mathbf{S}_{[t]}^r & = \underbrace{\left(\prod_{i=1}^r \left(\mathbf{I} - \beta_{[t]}^i \boldsymbol{k}_{[t]}^i \boldsymbol{k}_{[t]}^{i\top}\right) \brickred{\operatorname{Diag}(\boldsymbol{\alpha}_{[t]}^i)}\right)}_{:= \mathbf{P}_{[t]}^r} \cdot\mathbf{S}_{[t]}^{0} + \underbrace{\sum_{i=1}^{r} \left(\prod_{j=i+1}^r \left(\mathbf{I} - \beta_{[t]}^j \boldsymbol{k}_{[t]}^j \boldsymbol{k}_{[t]}^{j\top}\right)\brickred{\operatorname{Diag}(\boldsymbol{\alpha}_{[t]}^j)}\right)\cdot\beta_{[t]}^i \boldsymbol{k}_{[t]}^i\boldsymbol{v}_{[t]}^{i\top}}_{:=\mathbf{H}_{[t]}^r}
    \end{aligned}
    \label{eq:KDA-recurrent}
\end{equation}


\paragraph{WY Representation}shi
is typically employed to  pack a series rank-1 updates into a single compact representation \citep{bischof-wy-1987}. We follow the formulation of $\mathbf{P}$ in Comba \citep{hu2025comba} to reduce the need for an additional matrix inversion in subsequent computations.
    \begin{align}
        \mathbf{P}_{[t]}^r = \brickred{\operatorname{Diag}(\boldsymbol{\gamma}_{[t]}^r)} - \sum_{i=1}^{r} \brickred{\operatorname{Diag}(\boldsymbol{\gamma}_{[t]}^{i\rightarrow r})} \boldsymbol{k}_{[t]}^i \boldsymbol{w}_{[t]}^{i\top} &&
        \mathbf{H}_{[t]}^r = \sum_{i=1}^{t} \brickred{\operatorname{Diag}\left(\boldsymbol{\gamma}_{[t]}^{i\rightarrow r}\right)} \boldsymbol{k}_{[t]}^i \boldsymbol{u}_{[t]}^{i\top}
        \label{eq:PH_wy}
    \end{align}
    where the auxiliary vector $\boldsymbol{w}_t \in \mathbb{R}^{d_k}$ and $\boldsymbol{u}_t \in \mathbb{R}^{d_v}$ are computed via the following recurrence relation:
    \begin{align}
        \boldsymbol{w}_{[t]}^r &= \beta_{[t]}^r \left( \brickred{\operatorname{Diag}(\boldsymbol{\gamma}_{[t]}^r)} \boldsymbol{k}_{[t]}^r - \sum_{i=1}^{r-1} \boldsymbol{w}_{[t]}^i\left( \boldsymbol{k}_{[t]}^{i\top}\brickred{\operatorname{Diag}\left(\boldsymbol{\gamma}_{[t]}^{i\rightarrow r} \right)}\boldsymbol{k}_{[t]}^r \right)  \right) \\
        \boldsymbol{u}_{[t]}^r &= \beta_{[t]}^r \left(\boldsymbol{v}_{[t]}^r - \sum_{i=1}^{r-1}\boldsymbol{u}_{[t]}^i \left(\boldsymbol{k}_{[t]}^{i\top} \brickred{\operatorname{Diag}\left(\boldsymbol{\gamma}_{[t]}^{i\rightarrow r}\right)} \boldsymbol{k}_{[t]}^r\right)  \right)
    \end{align}


\paragraph{UT transform.}
We apply the UT transform \citep{joffrain-2006-ut} to reduce non-matmul FLOPs, which is crucial to enable better hardware utilization during training.
\begin{align}
\label{eq:gdn-wy}
\mathbf{M}_{[t]}&=\left(\mathbf{I} +  \operatorname{StrictTril} \left(\operatorname{Diag}\left(\beta_{[t]}\right) \left(\brickred{{\bm{\Gamma}}_{[t]}^{1\rightarrow C}} \odot \mathbf{K}_{[t]} \right) \left(\frac{\mathbf{K}_{[t]}}{\brickred{\bm{\Gamma}_{[t]}^{1\rightarrow C}}} \right)^\top\right) \right)^{-1} \operatorname{Diag}\left(\beta_{[t]}\right)\\
\mathbf{W}_{[t]} &= \mathbf{M}_{[t]} \left(\brickred{{\bm{\Gamma}}_{[t]}^{1\rightarrow C}}\odot\mathbf{K}_{[t]}\right),  \quad\quad\quad \mathbf{U}_{[t]}=\mathbf{M}_{[t]} \mathbf{V}_{[t]} 
\end{align}
The inverse of a lower triangular matrix can be efficiently computed through an iterative row-wise approach by forward substitution in Gaussian elimination \citep{grcar_2011}.

Equivalently, in matrix form, we can update the state in chunk-wise:
\begin{equation}
    \mathbf{S}_{[t+1]} = \brickred{\operatorname{Diag}(\boldsymbol{\gamma}_{[t]}^C)} \mathbf{S}_{[t]} +   \left(\brickred{\bm{\Gamma}_{[t]}^{i\rightarrow C}} \odot \mathbf{K}_{[t]}\right)^\top \left(\mathbf{U}_{[t]} - \mathbf{W}_{[t]} \mathbf{S}_{[t]}\right) \in \mathbb{R}^{d_k\times d_v}
\end{equation}
\definecolor{kvcolor}{RGB}{241,140,74}
\definecolor{americanrose}{rgb}{1.0, 0.01, 0.24}
\definecolor{bubblegum}{rgb}{0.99, 0.76, 0.8}
\definecolor{cadmiumred}{rgb}{0.89, 0.0, 0.13}
\begin{figure*}[h]
    \centering
    \begin{adjustbox}{width=0.6\textwidth,center}
    \centering
        \begin{tikzpicture}
            \tikzset{
                state/.style={
                    matrix of nodes,
                    nodes={
                        minimum size=10pt,
                        anchor=center,
                        inner sep=0pt,
                        font=\tiny
                    },
                    nodes in empty cells,
                    inner sep=0pt,
                    rounded corners=1pt
                },
                statelink/.style={
                    dash pattern={on 4pt off 4pt},
                    ->,
                    >={Straight Barb[length=5pt, width=6pt]},
                    line width=1.3pt,
                    shorten >=8pt,
                    shorten <=8pt,
                },
                seplink/.style={
                    dash pattern={on 4pt off 4pt},
                    line width=1pt,
                    shorten >=8pt,
                    shorten <=8pt,
                },
                kvlink/.style={
                    ->,
                    >={Straight Barb[length=5pt, width=6pt]},
                    line width=1.3pt,
                    shorten >=8pt,
                    shorten <=8pt,
                },
            }
            \matrix[state, anchor=west, left delimiter={[}, right delimiter={]}] (stp1) {
            |[fill=white]| {} & |[fill=midnightblue!20]| {} & |[fill=midnightblue!10]| {} & |[fill=midnightblue!20]| {} \\
            |[fill=midnightblue!60]| {} & |[fill=midnightblue!70]| {} & |[fill=white]| {} & |[fill=midnightblue!40]| {} \\
            |[fill=white]| {} & |[fill=midnightblue!30]| {} & |[fill=midnightblue!50]| {} & |[fill=white]| {} \\
            |[fill=midnightblue!50]| {} & |[fill=midnightblue!20]| {} & |[fill=midnightblue!20]| {} & |[fill=midnightblue!50]| {} \\
            };
            \node[right=10pt of stp1] (eq) {\large $=$};
            \matrix[state, right=10pt of eq.east, left delimiter={[}, right delimiter={]}] (decay) {
            |[fill=brickred!80]| {} & |[fill=white]| {} & |[fill=white]| {} & |[fill=white]| {} \\
            |[fill=white]| {} & |[fill=brickred!70]| {} & |[fill=white]| {} & |[fill=white]| {} \\
            |[fill=white]| {} & |[fill=white]| {} & |[fill=brickred!50]| {} & |[fill=white]| {} \\
            |[fill=white]| {} & |[fill=white]| {} & |[fill=white]| {} & |[fill=brickred!60]| {} \\
            };
            \matrix[state, anchor=west, right=20pt of decay, left delimiter={[}, right delimiter={]}] (st) {
            |[fill=midnightblue!20]| {} & |[fill=midnightblue!40]| {} & |[fill=midnightblue!60]| {} & |[fill=white]| {} \\
            |[fill=midnightblue!90]| {} & |[fill=midnightblue!70]| {} & |[fill=midnightblue!20]| {} & |[fill=midnightblue!70]| {} \\
            |[fill=midnightblue!30]| {} & |[fill=midnightblue!40]| {} & |[fill=midnightblue!60]| {} & |[fill=midnightblue!50]| {} \\
            |[fill=midnightblue!10]| {} & |[fill=white]| {} & |[fill=midnightblue!80]| {} & |[fill=midnightblue!20]| {} \\
            };
            \node[right=10pt of st] (plus) {\large $+$};
            \node[right=0pt of plus.east] (leftbracket) {\large $\Biggl($};
            \matrix[state,
            right=3pt of leftbracket
            ]
            (decay1)  {
            |[fill=brickred!80]| {} &|[fill=brickred!50]| {} &|[fill=brickred!40]| {} \\
            |[fill=brickred!70]| {} &|[fill=brickred!70]| {} &|[fill=brickred!80]| {} \\
            |[fill=brickred!50]| {} &|[fill=brickred!90]| {} &|[fill=brickred!60]| {} \\
            |[fill=brickred!60]| {} &|[fill=brickred!80]| {} &|[fill=brickred!90]| {} \\
            };
            \node[anchor=west, inner sep=2pt] (odot) at (decay1.east) {\large $\odot$};
            \matrix[state,
            anchor=west,
            right=0pt of odot
            ]
            (kv)  {
            |[fill=kvcolor!30]| {} &|[fill=kvcolor!30]| {} &|[fill=kvcolor!30]| {} \\
            |[fill=kvcolor!50]| {} &|[fill=kvcolor!50]| {} &|[fill=kvcolor!50]| {} \\
            |[fill=kvcolor!70]| {} &|[fill=kvcolor!70]| {} &|[fill=kvcolor!70]| {} \\
            |[fill=kvcolor!90]| {} &|[fill=kvcolor!90]| {} &|[fill=kvcolor!90]| {} \\
            };
            \node[right=3pt of kv] (rightbracket) {\large $\Biggl)$};
            \matrix[state,
            anchor=west, 
            ]
            (kvt) at ($(kv.east)+(20pt,0)$) {
            |[fill=kvcolor!30]| {} & |[fill=kvcolor!50]| {} & |[fill=kvcolor!70]| {} & |[fill=kvcolor!90]| {} \\
            |[fill=kvcolor!30]| {} & |[fill=kvcolor!50]| {} & |[fill=kvcolor!70]| {} & |[fill=kvcolor!90]| {} \\
            |[fill=kvcolor!30]| {} & |[fill=kvcolor!50]| {} & |[fill=kvcolor!70]| {} & |[fill=kvcolor!90]| {} \\
            };
            

            
            
            
        \end{tikzpicture}
        \end{adjustbox}

    \captionsetup{labelformat=empty,labelsep=none}
    \label{fig:KDA-chunks}
\end{figure*}


During the output stage, we adopt an inter-block recurrent and intra-block parallel strategy to maximize matrix multiplication throughput, thereby fully utilizing the computational potential of Tensor Cores.
\begin{equation}
\label{eq:gdn-o}
    \mathbf{O}_{[t]} = 
    \underbrace{\left(\brickred{{\bm{\Gamma}}_{[t]}^{1\rightarrow C}} \odot\mathbf{Q}_{[t]}\right)
    \mathbf{S}_{[t]}}_\text{inter chunk} + \underbrace{\operatorname{Tril}\left(\left(\brickred{{\bm{\Gamma}}_{[t]}^{1\rightarrow C}} \odot \mathbf{Q}_{[t]} \right) \left(\frac{\mathbf{K}_{[t]}}{\brickred{{\bm{\Gamma}}_{[t]}^{1\rightarrow C}}} \right)^\top \right)}_\text{intra chunk} \underbrace{\left(\mathbf{U}_{[t]} - \mathbf{W}_{[t]} \mathbf{S}_{[t]}\right)}_{\text{``pseudo''-value term}} \in \mathbb{R}^{C\times d_v}
\end{equation}
\definecolor{kvcolor}{RGB}{241,140,74}
\definecolor{americanrose}{rgb}{1.0, 0.01, 0.24}
\definecolor{bubblegum}{rgb}{0.99, 0.76, 0.8}
\definecolor{cadmiumred}{rgb}{0.89, 0.0, 0.13}
\begin{figure*}[h]
    \centering
    \begin{adjustbox}{width=0.6\textwidth,center}
    \centering
        \begin{tikzpicture}
            \tikzset{
                state/.style={
                    matrix of nodes,
                    nodes={
                        minimum size=10pt,
                        anchor=center,
                        inner sep=0pt,
                        font=\tiny
                    },
                    nodes in empty cells,
                    inner sep=0pt,
                    rounded corners=1pt
                },
                statelink/.style={
                    dash pattern={on 4pt off 4pt},
                    ->,
                    >={Straight Barb[length=5pt, width=6pt]},
                    line width=1.3pt,
                    shorten >=8pt,
                    shorten <=8pt,
                },
                seplink/.style={
                    dash pattern={on 4pt off 4pt},
                    line width=1pt,
                    shorten >=8pt,
                    shorten <=8pt,
                },
                kvlink/.style={
                    ->,
                    >={Straight Barb[length=5pt, width=6pt]},
                    line width=1.3pt,
                    shorten >=8pt,
                    shorten <=8pt,
                },
            }
            
            \matrix[state,
            anchor=north, 
            ]
            (o) at ($(stp1.south)-(0,45pt)$) {
            |[fill=kvcolor!30]| {} & |[fill=kvcolor!50]| {} & |[fill=kvcolor!70]| {} & |[fill=kvcolor!90]| {} \\
            |[fill=kvcolor!30]| {} & |[fill=kvcolor!50]| {} & |[fill=kvcolor!70]| {} & |[fill=kvcolor!90]| {} \\
            |[fill=kvcolor!30]| {} & |[fill=kvcolor!50]| {} & |[fill=kvcolor!70]| {} & |[fill=kvcolor!90]| {} \\
            };
            \node[] at (o-|eq) (eq1) {\large $=$};

            \node[right=0pt of eq1.east] (leftbracket) {\large $\Biggl($};
            \matrix[state,
            right=3pt of leftbracket
            ]
            (decay1)  {
            |[fill=brickred!80]| {} &|[fill=brickred!70]| {} &|[fill=brickred!50]| {} &|[fill=brickred!60]| {} \\
            |[fill=brickred!50]| {} &|[fill=brickred!70]| {} &|[fill=brickred!90]| {} &|[fill=brickred!80]| {} \\
            |[fill=brickred!40]| {} &|[fill=brickred!80]| {} &|[fill=brickred!60]| {} &|[fill=brickred!90]| {} \\
            };
            \node[anchor=west, inner sep=2pt] (odot) at (decay1.east) {\large $\odot$};
            \matrix[state,
            anchor=west,
            right=0pt of odot
            ]
            (q)  {
            |[fill=kvcolor!30]| {} &|[fill=kvcolor!50]| {} &|[fill=kvcolor!70]| {} &|[fill=kvcolor!90]| {} \\
            |[fill=kvcolor!30]| {} &|[fill=kvcolor!50]| {} &|[fill=kvcolor!70]| {} &|[fill=kvcolor!90]| {} \\
            |[fill=kvcolor!30]| {} &|[fill=kvcolor!50]| {} &|[fill=kvcolor!70]| {} &|[fill=kvcolor!90]| {} \\
            };
            \node[right=3pt of q] (rightbracket) {\large $\Biggl)$};
            \matrix[state, right=5pt of rightbracket, left delimiter={[}, right delimiter={]}] (stp1) {
            |[fill=white]| {} & |[fill=midnightblue!20]| {} & |[fill=midnightblue!10]| {} & |[fill=midnightblue!20]| {} \\
            |[fill=midnightblue!60]| {} & |[fill=midnightblue!70]| {} & |[fill=white]| {} & |[fill=midnightblue!40]| {} \\
            |[fill=white]| {} & |[fill=midnightblue!30]| {} & |[fill=midnightblue!50]| {} & |[fill=white]| {} \\
            |[fill=midnightblue!50]| {} & |[fill=midnightblue!20]| {} & |[fill=midnightblue!20]| {} & |[fill=midnightblue!50]| {} \\
            };
            \node[right=10pt of stp1] (plus) {\large $+$};
            
            \matrix[state, right=10pt of plus, left delimiter={[}, right delimiter={]}] (attn) {
            |[fill=cadmiumred!70]| {} & |[fill=white]| {} & |[fill=white]| {} \\
            |[fill=cadmiumred!40]| {} & |[fill=cadmiumred!80]| {} & |[fill=white]| {} \\
            |[fill=cadmiumred!50]| {} & |[fill=cadmiumred!30]| {} & |[fill=cadmiumred!60]| {} \\
            };
            \matrix[state,
            anchor=west, 
            ]
            (kvt) at ($(attn.east)+(10pt,0)$) {
            |[fill=kvcolor!30]| {} & |[fill=kvcolor!50]| {} & |[fill=kvcolor!70]| {} & |[fill=kvcolor!90]| {} \\
            |[fill=kvcolor!30]| {} & |[fill=kvcolor!50]| {} & |[fill=kvcolor!70]| {} & |[fill=kvcolor!90]| {} \\
            |[fill=kvcolor!30]| {} & |[fill=kvcolor!50]| {} & |[fill=kvcolor!70]| {} & |[fill=kvcolor!90]| {} \\
            };
            
            
        \end{tikzpicture}
        \end{adjustbox}

    \captionsetup{labelformat=empty,labelsep=none}
    \label{fig:KDA-chunko}
\end{figure*}





\subsection{Efficiency Analysis}
\begin{wrapfigure}[14]{r}{0.34\textwidth}
    \centering
    \vspace{-2em}
    \resizebox{\linewidth}{!}{
        \begin{tikzpicture}
            \begin{axis}[
                    trim axis left,
                    trim axis right,
                    ymajorgrids=true,
                    xmajorgrids=true,
                    tickwidth=0pt,
                    tick align=inside,
                    xlabel=Input length,
                    enlarge x limits=0.15,
                    width=10cm, height=7.5cm,
                    ymin=0, ymax=64,
                    symbolic x coords={
                            2K,4K,8K,16K,32K,64K
                        },
                    ytick={0,16,32,48,64},
                    yticklabels={0,16,32,48,64},
                    xlabel near ticks,
                    ylabel=Execution Time (ms),
                    ylabel style={at={(0.05,0.5)}},
                    axis line style={opacity=0},
                    legend style={
                            at={(0,1)},
                            anchor=north west,
                            legend cell align=left,
                            font=\small,
                        },
                ]


                \addplot[
                    line width=1.5pt,
                    dashdotdotted,
                    mark=star,
                    mark size=2pt,
                    mark options={scale=1},
                    color=darkcyan
                ] plot coordinates {
                        (2K,  2.021408 )
                        (4K,  3.900816 )
                        (8K,  7.550000 )
                        (16K, 14.894096)
                        (32K, 29.636353)
                        (64K, 59.221313)
                    };
                \addlegendentry{DPLR}

                \addplot[
                    line width=1.5pt,
                    mark=pentagon*,
                    draw=blue!60,,
                    mark options={
                            fill=blue!60,
                            fill opacity=1.0,
                            solid
                        },
                    mark size=1.5pt,
                    opacity=1.0,
                ] plot coordinates {
                        (2K,    1.698688)
                        (4K,    2.026816)
                        (8K,    3.877184)
                        (16K,   7.599360)
                        (32K,  14.811120)
                        (64K,  29.831713)
                    };
                \addlegendentry{\text{KDA (ours)}}




                \draw [semithick, black] (axis description cs:0,0) -- (axis description cs:1,0); % This line adds a thick line at y=0
                \draw [semithick, black] (axis description cs:0,1) -- (axis description cs:1,1); % This line adds a thick line at y=1


            \end{axis}

        \end{tikzpicture}
    }

    \caption{
        Execution time of kernels for varying input lengths, with a uniform batch size of 1 and 16 heads.
    }
    \label{fig:kernel}
\end{wrapfigure}
In terms of representational capacity, KDA aligns with the generalized DPLR formulation, i.e., $\mathbf{S}_t = (\mathbf{D} - \bm{a}_t \bm{b}_t^{\top}) \mathbf{S}_{t-1} + \bm{k}_t \bm{v}_t^{\top}$, both exhibiting fine-grained decay behavior. However, such fine-grained decay introduces numerical precision issues during division operations (e.g., the intra-chunk computation in Eq.~\ref{eq:gdn-o}). To address this, prior work such as GLA~\citep{yang-etal-2024-gla} performs computations in the logarithmic domain and introduces secondary chunking in full precision. This approach, however, prevents full utilization of half-precision matrix multiplications and significantly reduces operator speed.
By binding both variables $\bm a$ and $\bm b$ to $\bm k$, KDA effectively alleviates this bottleneck—reducing the number of second-level chunk matrix computations from four to two, and further eliminating three additional matrix multiplications. As a result, the operator efficiency of KDA improves by roughly 100\% compared to the DPLR formulation. A detailed analysis is provided in \S \ref{sec:related_dplr}.







































