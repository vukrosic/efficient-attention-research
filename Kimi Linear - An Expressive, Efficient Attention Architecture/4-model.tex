

% \pgfplotsset{every tick label/.append style={font=\footnotesize}}
% \pgfplotsset{compat=1.14}



\definecolor{fgate_color}{RGB}{252,224,225}
\definecolor{delta_color}{RGB}{242,243,193}
\definecolor{swa_color}{RGB}{252,224,225}
\definecolor{add_norm_color}{RGB}{252,226,187}
\definecolor{glu_color}{RGB}{194,232,247}
\definecolor{silu_color}{RGB}{203,231,207}
\definecolor{linear_color}{RGB}{220,223,240}
\definecolor{conv_color}{RGB}{252,224,225}
\definecolor{l2_color}{RGB}{252,226,187}
\definecolor{gray_bbox_color}{RGB}{243,243,244}
\definecolor{oproj_color}{RGB}{220,223,240}
\definecolor{operator_color}{RGB}{252,224,225}
\begin{figure}[t!]
\centering
\scalebox{0.85}{
\tikzset{
model/.style={
    draw=black,
    very thick,
    % ultra thick,
    % line width=1.3pt,
    fill=gray_bbox_color,
    minimum width=118pt,
    rounded corners=10pt
  },
gdelta/.style={
    draw=black,
    very thick,
    % ultra thick,
    % line width=1.3pt,
    fill=gray_bbox_color,
    % fill=delta_color,
    minimum width=200pt,
    minimum height=200pt,
    rounded corners=10pt
  },
tokenmixer/.style={
    draw=black,
    very thick,
    % ultra thick,
    % line width=1.3pt,
    fill=delta_color!80,
    minimum width=78pt,
    minimum height=0.7cm,
    rounded corners=3pt
  },
swa/.style={
    draw=black,
    very thick,
    % ultra thick,
    % line width=1.3pt,
    fill=swa_color!80,
    minimum width=78pt,
    minimum height=0.7cm,
    rounded corners=3pt
  },
glu/.style={
    draw=black,
    very thick,
    % ultra thick,
    % line width=1.3pt,
    fill=glu_color!80,
    minimum width=78pt,
    minimum height=0.7cm,
    rounded corners=3pt
  },
norm/.style={
    draw=black,
    very thick,
    fill=add_norm_color!80,
    minimum width=40pt,
    % minimum height=1cm,
    rounded corners=3pt,
    align=center,
  },
linear/.style={
    draw=black,
    very thick,
    % ultra thick,
    % line width=1.3pt,
    fill=oproj_color!80,
    minimum width=40pt,
    % minimum height=1cm,
    rounded corners=3pt
  },
stacked/.style={
    draw=black,
    very thick,
    % ultra thick,
    % line width=1.3pt,
    fill=linear_color!80,
    minimum width=40pt,
    minimum height=15pt,
    rounded corners=3pt,
    % trapezium, trapezium angle=100, draw,inner xsep=2pt,outer sep=0pt,
  },
    % Modified trapezoid style with rounded corners
    trapezoid/.style={
        trapezium,
        trapezium left angle=110,
        trapezium right angle=110,
        rounded corners=3pt,
        inner xsep=1pt,
        outer sep=0pt,
    },
conv/.style={
    draw=black,
    very thick,
    minimum width=30pt,
    fill=conv_color!80,
    rounded corners=3pt,
    rectangle,
    font=\small
  },
l2/.style={
    draw=black,
    very thick,
    minimum width=30pt,
    fill=l2_color!80,
    rounded corners=3pt,
    rectangle,
    font=\small
  },
fgate/.style={
    draw=black,
    very thick,
    % ultra thick,
    % line width=1.3pt,
    fill=fgate_color!80,
    minimum width=30pt,
    minimum height=15pt,
    rounded corners=3pt
  },
oproj/.style={
    draw=black,
    very thick,
    % ultra thick,
    % line width=1.3pt,
    fill=oproj_color!80,
    minimum width=78pt,
    % minimum height=1cm,
    rounded corners=3pt
  },
silu/.style={
    draw=black,
    very thick,
    % ultra thick,
    % line width=1.3pt,
    fill=silu_color,
    minimum width=1.1cm,
    % minimum height=1cm,
    rounded corners=3pt
  },
layerlink/.style={
    -latex,
    very thick,
    % shorten >=1pt,
    % shorten <=1pt
  },
modulelink/.style={
    -latex,
    very thick,
    densely dashed,
    shorten >=1pt,
    shorten <=1pt,
    rounded corners=3pt
  },
normlink/.style={
    very thick,
    % shorten >=1pt,
    % shorten <=1pt
  },
residual/.style={
    % -latex,
    very thick,
    rounded corners=5pt
  },
qf/.style={
    -latex,
    very thick,
    rounded corners=5pt
  },
oplus/.style={
    draw=black,
    % thick,
    line width=1pt,
    circle,
    minimum size=8pt,
    inner sep=0pt,
    outer sep=0pt,
    path picture={
        \draw (path picture bounding box.center) -- ++(0.3cm,0)
        (path picture bounding box.center) -- ++(-0.3cm,0)
        (path picture bounding box.center) -- ++(0,0.3cm)
        (path picture bounding box.center) -- ++(0,-0.3cm);
      },
  },
otimes/.style={
    draw=black,
    very thick,
    circle,
    minimum size=8pt,
    inner sep=0pt,
    outer sep=0pt,
    path picture={
        \draw (path picture bounding box.center) -- ++(0.25cm,0.25cm)
        (path picture bounding box.center) -- ++(-0.25cm,-0.25cm)
        (path picture bounding box.center) -- ++(-0.25cm,0.25cm)
        (path picture bounding box.center) -- ++(0.25cm,-0.25cm);
      }
  },
sigmoid/.style={
    draw=black,
    thick,
    line width=1pt,
    circle,
    minimum size=8pt,
    inner sep=0pt,
    outer sep=0pt,
    path picture={
        \draw[domain=-1.5:0, samples=50, variable=\x, blue, thick]
        plot ({\x}, {0});
        \draw[domain=0:1.5, samples=50, variable=\x, blue, thick]
        plot ({\x}, {\x});
      }
  },
swish/.style={
    draw=black,
    thick,
    line width=1pt,
    circle,
    minimum size=8pt,
    inner sep=0pt,
    outer sep=0pt,
    path picture={
        \draw[domain=-1.5:0, samples=50, variable=\x, blue, thick]
        plot ({\x}, {0});
        \draw[domain=0:1.5, samples=50, variable=\x, blue, thick]
        plot ({\x}, {\x});
      }
  },
kgdelta/.style={
    draw=black,
    very thick,
    % ultra thick,
    % line width=1.3pt,
    fill=delta_color!50,
    minimum width=115pt,
    minimum height=0.8cm,
    rounded corners=3pt
  },
}
\resizebox{\textwidth}{!}{
\begin{tikzpicture}
\centering

%%%%%%%%%%%%%%%%%%%%%%%%%%%%%%%%%%%%%%%%%%%%%%%%%%%%%%%%%%%%%%%%%%%%%%%%%%%%%%%%%%%%%%%%
% GATED DELTANET H1
%%%%%%%%%%%%%%%%%%%%%%%%%%%%%%%%%%%%%%%%%%%%%%%%%%%%%%%%%%%%%%%%%%%%%%%%%%%%%%%%%%%%%%%%

\node[model, minimum height=165pt] (model) at (0,0) {};
\node[anchor=east,xshift=-2pt] at (model.west) (ntimes) {$N\times$};
\node[tokenmixer, anchor=south, yshift=20pt] at (model.south) (gdelta1) {Gated DeltaNet};
\draw[layerlink] ($(model.south) + (0,-12pt)$) -- (gdelta1.south);
\node[glu, anchor=south, yshift=8pt] at (gdelta1.north) (mlp1) {MLP};
\draw[normlink] (gdelta1.north) -- (mlp1.south);
\node[oplus, anchor=south, yshift=4pt] at (mlp1.north) (oplus1) {};

\node[swa, anchor=south, yshift=12pt] at (oplus1.north) (swa) {SWA};
\draw[normlink] (mlp1.north) -- (oplus1.south);
\draw[layerlink] (oplus1.north) -- (swa.south);
\draw[residual] ([yshift=-10pt]gdelta1.south) -- ([xshift=-48pt,yshift=-10pt]gdelta1.south) -- ([xshift=-48pt]oplus1.center) -- (oplus1.center);
\node[glu, anchor=south, yshift=8pt] at (swa.north) (mlp2) {MLP};
\draw[normlink] (swa.north) -- (mlp2.south);
\node[oplus, anchor=south, yshift=4pt] at (mlp2.north) (oplus2) {};

\draw[normlink] (mlp2.north) -- (oplus2.south);
\draw[residual] ([xshift=0pt,yshift=6pt]oplus1.center) -- ([xshift=-48pt,yshift=6pt]oplus1.center) -- ([xshift=-48pt]oplus2.center) -- (oplus2.center);
% \node[linear, anchor=south, yshift=5pt] at (norm3.north) (linear) {Linear};
% \draw[layerlink] (oplus2.north) -- (norm3.south);
% \draw[normlink] (norm3.north) -- (linear.south);
\node[above=12pt] at (model.north) (output) {\textcolor{white}{Outputs}};
\draw[layerlink] (oplus2.north) -- (output.south);


%%%%%%%%%%%%%%%%%%%%%%%%%%%%%%%%%%%%%%%%%%%%%%%%%%%%%%%%%%%%%%%%%%%%%%%%%%%%%%%%%%%%%%%%
% GATED DELTANET H2
%%%%%%%%%%%%%%%%%%%%%%%%%%%%%%%%%%%%%%%%%%%%%%%%%%%%%%%%%%%%%%%%%%%%%%%%%%%%%%%%%%%%%%%%

\node[model, minimum height=240pt, anchor=west, xshift=20pt] (model1) at (model.east) {};
% \node[anchor=north, align=center] at (1.250000,-1.500000) {Inputs};
\node[below=12pt] at (model1.south) (input) {Gated DeltaNet-H2};
\node[] at (model |- input) (input0) {Gated DeltaNet-H1};

\node[tokenmixer, anchor=south, yshift=20pt] at (model1.south) (mamba2) {Mamba2};
\draw[layerlink] (input.north) -- (mamba2.south);
\node[glu, anchor=south, yshift=8pt] at (mamba2.north) (mlp1) {MLP};
\draw[normlink] (mamba2.north) -- (mlp1.south);
\node[oplus, anchor=south, yshift=4pt] at (mlp1.north) (oplus1) {};

\node[tokenmixer, anchor=south, yshift=12pt] at (oplus1.north) (gdelta2) {Gated DeltaNet};
\draw[normlink] (mlp1.north) -- (oplus1.south);
\draw[layerlink] (oplus1.north) -- (gdelta2.south);
\draw[residual] ([yshift=-10pt]mamba2.south) -- ([xshift=-48pt,yshift=-10pt]mamba2.south) -- ([xshift=-48pt]oplus1.center) -- (oplus1.center);
\node[glu, anchor=south, yshift=8pt] at (gdelta2.north) (mlp2) {MLP};
\draw[normlink] (gdelta2.north) -- (mlp2.south);
\node[oplus, anchor=south, yshift=4pt] at (mlp2.north) (oplus2) {};

\node[swa, anchor=south, yshift=12pt] at (oplus2.north) (swa) {SWA};
\draw[normlink] (mlp2.north) -- (oplus2.south);
\draw[layerlink] (oplus2.north) -- (swa.south);
\draw[residual] ([yshift=-10pt]gdelta2.south) -- ([xshift=-48pt,yshift=-10pt]gdelta2.south) -- ([xshift=-48pt]oplus2.center) -- (oplus2.center);
\node[glu, anchor=south, yshift=8pt] at (swa.north) (mlp3) {MLP};
\draw[normlink] (swa.north) -- (mlp3.south);
\node[oplus, anchor=south, yshift=4pt] at (mlp3.north) (oplus3) {};
  
\draw[normlink] (mlp3.north) -- (oplus3.south);
\draw[residual] ([xshift=0pt,yshift=6pt]oplus2.center) -- ([xshift=-48pt,yshift=6pt]oplus2.center) -- ([xshift=-48pt]oplus3.center) -- (oplus3.center);
% \node[linear, anchor=south, yshift=5pt] at (norm3.north) (linear) {Linear};
% \draw[layerlink] (oplus2.north) -- (norm3.south);
% \draw[normlink] (norm3.north) -- (linear.south);
\node[above=12pt] at (model1.north) (output) {\textcolor{white}{Outputs}};
\draw[layerlink] (oplus3.north) -- (output.south);


%%%%%%%%%%%%%%%%%%%%%%%%%%%%%%%%%%%%%%%%%%%%%%%%%%%%%%%%%%%%%%%%%%%%%%%%%%%%%%%%%%%%%%%%
% BLOCK DESIGN
%%%%%%%%%%%%%%%%%%%%%%%%%%%%%%%%%%%%%%%%%%%%%%%%%%%%%%%%%%%%%%%%%%%%%%%%%%%%%%%%%%%%%%%%

\node[gdelta, anchor=west,xshift=30pt, minimum width=220pt] (gdelta) at (model1.east)  {};
\node[] at (gdelta |- input) (input1) {Block Design};


\node[kgdelta, yshift=15pt,minimum width=150pt] at (gdelta.mid) (kernel) {Gated Delta Rule };
\node[below=12pt] at (gdelta.south) (input) {\textcolor{white}{Inputs}};

\node[stacked, trapezium, trapezium left angle=110, trapezium right angle=110, inner xsep=1pt, outer sep=0pt, minimum width=30pt, anchor=south, yshift=20pt] at (gdelta.south) (vproj) {Linear};
\node[conv, anchor=south, yshift=8pt] at (vproj.north) (vconv) {Conv};
\node[swish, anchor=south, yshift=4pt] at (vconv.north) (vsilu) {};
\draw[layerlink] (vsilu.north) -- (vsilu|-kernel.south) node[pos=0.8, right] {$\boldsymbol{v}$};
\draw[normlink] (vsilu.south) -- (vconv.north);
\draw[normlink] (vconv.south) -- (vproj.north);
\draw[residual] (input.north) -- (gdelta.south) -- (vproj.south);

\node[stacked, trapezium, trapezium left angle=110, trapezium right angle=110, inner xsep=1pt, outer sep=0pt, minimum width=30pt, anchor=south, yshift=19pt, opacity=0.2] at ($(gdelta.south west)!0.27!(gdelta.south east)$) (kproj) {\textcolor{white}{Linear}};
\node[conv, anchor=south, yshift=8pt, opacity=0.2] at (kproj.north) (kconv) {\textcolor{white}{Conv}};
\node[swish, anchor=south, yshift=4pt, opacity=0.2] at (kconv.north) (ksilu) {};
\node[l2, anchor=south, yshift=4pt, opacity=0.2] at (ksilu.north) (kl2) {\textcolor{white}{L2}};
\draw[layerlink, opacity=0.2] (kl2.north) -- (kl2|-kernel.south);
\node[right] at ($(kl2)+(0,20pt)$) {$\boldsymbol{k}$};
\draw[normlink, opacity=0.2] (ksilu.north) -- (kl2.south);
\draw[normlink, opacity=0.2] (ksilu.south) -- (kconv.north);
\draw[normlink, opacity=0.2] (kconv.south) -- (kproj.north);
\draw[residual, opacity=0.2] (input.north) -- (gdelta.south) |- ([xshift=-10pt,yshift=10pt]gdelta.south) -| (kproj.south);

\node[stacked, trapezium, trapezium left angle=110, trapezium right angle=110, inner xsep=1pt, outer sep=0pt, minimum width=30pt, anchor=south, yshift=21pt] at ($(gdelta.south west)!0.24!(gdelta.south east)$) (qproj) {Linear};
\node[conv, anchor=south, yshift=8pt] at (qproj.north) (qconv) {Conv};
\node[swish, anchor=south, yshift=4pt] at (qconv.north) (qsilu) {};
\node[l2, anchor=south, yshift=4pt] at (qsilu.north) (ql2) {L2};
\draw[layerlink] (ql2) -- (ql2|-kernel.south) node[pos=0.55, left] {$\vq$};
\draw[normlink] (qsilu.north) -- (ql2.south);
\draw[normlink] (qsilu.south) -- (qconv.north);
\draw[normlink] (qconv.south) -- (qproj.north);
\draw[residual] (input.north) -- (gdelta.south) |- ([xshift=-10pt,yshift=10pt]gdelta.south) -| (qproj.south);


% \node[swish, xshift=-40pt, yshift=50pt] at ($(gdelta.south west)!0.375!(gdelta.south east)$) (qsilu) {};
% link between qsilu and qproj
% \draw[normlink] (qsilu.south) -- (qsilu|-qproj.north);

\node[stacked, isosceles triangle, isosceles triangle apex angle=105, draw=black, very thick,  inner sep=.75pt, anchor=south, yshift=22pt, shape border rotate=90] at ($(gdelta.south west)!0.68!(gdelta.south east)$) (fproj) {Lin.};
\node[stacked,isosceles triangle, isosceles triangle apex angle=105, draw=black, very thick, fill=fgate_color!80, inner sep=.9pt, anchor=south, yshift=20pt, shape border rotate=90, opacity=0.2] at ($(gdelta.south west)!0.71!(gdelta.south east)$) (bproj) {\textcolor{white}{Lin.}};
% \node[anchor=center] at ($(fproj.south)!0.4!(fproj.north)$) {\footnotesize Linear};
\draw[qf] ($(fproj.north)-(0,2pt)$) -- (fproj.north|-kernel.south) node[pos=0.89, left] {${\alpha}$};
\draw[qf, opacity=0.2] ($(bproj.north)-(0,2pt)$) -- (bproj.north|-kernel.south) node[opacity=1, pos=0.89, right] {${\beta}$};
\draw[residual, opacity=0.2] (input.north) -- (gdelta.south) |- ([xshift=-10pt,yshift=10pt]gdelta.south) -| (bproj.south);

\draw[residual] (input.north) -- (gdelta.south) |- ([xshift=10pt,yshift=10pt]gdelta.south) -| (fproj.south);

\node[stacked, trapezium, trapezium left angle=110, trapezium right angle=110, inner xsep=1pt, outer sep=0pt, minimum width=30pt, anchor=south, yshift=20pt] at ($(gdelta.south west)!0.89!(gdelta.south east)$) (gproj) {Linear};
\draw[residual] (input.north) -- (gdelta.south) |- ([xshift=10pt,yshift=10pt]gdelta.south) -| (gproj.south);
\node[sigmoid, yshift=20pt] at (gproj.north) (sigmoid) {};
\draw[normlink] (gproj.north) -- (sigmoid);

\node[norm, anchor=south, yshift=10pt] at (kernel.north) (norm4) {Norm};
\node[otimes, anchor=south, yshift=6pt] at (norm4.north) (otimes) {};
\node[draw=black, very thick, fill=oproj_color!80, minimum width=78pt, minimum height=10pt, shape=trapezium, trapezium angle=45, rounded corners=3pt, anchor=south, yshift=6pt] at (otimes.north) (oproj) {Linear};
\draw[layerlink] (kernel.north) -- (norm4.south);
\draw[normlink] (norm4.north) -- (otimes.south);
\draw[normlink] (otimes.north) -- (oproj.south);
\draw[residual] (sigmoid) |- (otimes) node[pos=0.8, below] {};

\node[above=12pt] at (gdelta.north) (output) {\textcolor{white}{Outputs}};
\draw[layerlink] (oproj.north) -- (output.south);
\draw[modulelink] (gdelta2.east) -| ([yshift=-10pt]$(model1.east)!0.5!(gdelta.west)$) |- (gdelta.west);
% \draw[] (mlp1.south east) -- ($(gdelta.south west) + (-1pt, 5pt)$);
% \draw[] (mlp1.north east) -- ($(gdelta.north west) + (-1pt, -5pt)$);
\end{tikzpicture}
}
}
\caption{\small
Visualization of the (hybrid) architecture and block design of Gated DeltaNet models. Gated DeltaNet-H1 and H2 use Gated DeltaNet + SWA and Mamba2 + Gated DeltaNet + SWA patterns, respectively. In the block design, query/key paths consist of linear proj., shortconv., SiLU and L2 norm; value path includes linear proj., shortconv. and SiLU; alpha/beta use linear proj.; and output gate applies linear proj. with SiLU.
}
\label{fig:gated_deltanet_model}
\end{figure}


\section{The Kimi Linear Model Architecture}
The main backbone of our model architecture follows Moonlight \citep{liu-2025-moonlight}.
In addition to fine-grained gating, we also leverage several components to further improve the expressiveness of Kimi Linear.
The overall Kimi Linear architecture is shown in Figure \ref{fig:scaling-model}.

\paragraph{Neural Parameterization}
Let $\bm{x}_t \in \mathbb{R}^d$ be the $t$-th token input representation, the input to KDA for each
head $h$ is computed as follows
\begin{align*}
\bm{q}^h_t,\bm{k}^h_t &= \operatorname{L2Norm}(\operatorname{Swish}(\operatorname{ShortConv}(\mathbf{W}^h_{q/k}\bm{x}_t)))\in \mathbb{R}^{d_k}\\
\bm{v}^h_t &= \operatorname{Swish}(\operatorname{ShortConv}(\mathbf{W}^h_v\bm{x}_t))\in \mathbb{R}^{d_v} \\
\brickred{\bm{\alpha}^h_t} &= f(\mathbf{W}_{\alpha}^{\uparrow}\mathbf{W}_{\alpha}^{\downarrow}\bm{x}_t) \in [0,1]^{d_k}\\
\beta^h_t &= \operatorname{Sigmoid}(\mathbf{W}_{\beta}^h\bm{x}_t) \in [0,1]\\
\end{align*}
where $d_k, d_v$ represent the key and value head dimensions, which are set to 128 for all experiments.
For $\bm{q},\bm{k},\bm{v}$, we apply a $\operatorname{ShortConv}$ followed by a $\operatorname{Swish}$ activation, following \citep{yang-2025-gdn}. 
The $\bm{q}$ and $\bm{k}$ representations are further normalized using $\operatorname{L2Norm}$ to ensure eigenvalues stability, as suggested by \cite{yang-2024-parallelizing}. 
The per-channel decay $\brickred{\bm{\alpha}^h_t}$ is parameterized via a low-rank projection (\(\mathbf{W}_{\alpha}^{\downarrow}\) and \(\mathbf{W}_{\alpha}^{\uparrow}\) with rank equal to the head dimension) and a decay function $f(\cdot)$ similar to those used in GDN and Mamba~\citep{yang-2025-gdn,mamba2}. 
Before the output projection through $\mathbf{W}_o \in \mathbb{R}^{d \times d}$, we use a head-wise RMSNorm \citep{zhang2019root} and a data-dependent gating mechanism \citep{qiu2025gated} parameterized as:
\begin{equation}
\begin{aligned}
\bm{o}_t = \mathbf{W}_o\left( \operatorname{Sigmoid}\left(\mathbf{W}_g^{\uparrow}\mathbf{W}_g^{\downarrow} \bm{x}_t\right)\odot \operatorname{RMSNorm}\left(\operatorname{KDA}\left( \bm{q}_t,\bm{k}_t,\bm{v}_t,\brickred{\bm{\alpha}_t},\beta_t \right) \right)\right) 
\end{aligned}
\end{equation}
Here, the output gate adopts a low-rank parameterization similar to the forget gate, to ensure a fair parameter comparison, while maintaining performance comparable to full-rank gating and alleviating the Attention Sink \citep{qiu2025gated}. 
The choice of nonlinear activation function is further discussed in \S\ref{sec:ablation}.









\paragraph{Hybrid model architecture}
Long‑context retrieval remains the primary bottleneck for pure linear attention, we therefore hybridize KDA with a small number of full global‑attention (Full MLA) layers \cite{deepseekaiv3}. 
For Kimi Linear, we chose a layerwise approach (alternating entire layers) over a headwise one (mixing heads within layers) for its superior infrastructure simplicity and training stability. 
Empirically, a uniform 3:1 ratio, i.e., repeating 3 KDA layers to 1 full MLA layer, provided the best quality–throughput trade‑off.
We discuss other hybridization strategies in \S~\ref{sec:hybrid}.


\paragraph{No Position Encoding (NoPE) for MLA Layers.}
In Kimi Linear, we apply NoPE to all full attention (MLA) layers.
This design delegates the entire responsibility for encoding positional information and recency bias (see \S~\ref{sec:delta_rule}) to the KDA layers.
KDA is thus established as the primary position-aware operator, fulfilling a role analogous to, or arguably stronger than, auxiliary components like short convolutions \citep{allen2025physics} or SWA \citep{puvvada2025swangpt}.
Our findings align with prior results \citep{yang2025ropenopeagainnew,barbero2025round,deepseekaiv3}, who similarly demonstrated that complementing global NoPE attention with a dedicated position-aware mechanism yields competitive long-context performance.

We note that NoPE offers practical advantages, particularly for MLA. 
First, NoPE enables their conversion to the highly-efficient pure Multi-Query Attention (MQA) during inference. 
Second, it simplifies long-context training, as it obviates the need for RoPE parameter adjustments, such as frequency base tuning or methods like YaRN \citep{peng2023yarn}.
































